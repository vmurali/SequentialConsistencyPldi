\section{Specifying Sequential Consistency}\label{sec:sc}

Our final theorem in this paper establishes that a particular complex hardware
system implements sequential consistency (SC) properly.  We state the theorem
in terms of the trace refinement relation $\sqsubseteq$.
Therefore, we need to define an LTS embodying SC.  The simpler this
system, the better.  We do not need to worry about its performance properties,
since we will prove that an optimized system remains faithful to it.

\begin{figure}
\small
\centering
\begin{boxedminipage}[c]{.48\textwidth}
\inference
[Halt]
{\dec(s,\pc) = H}
{\io{P$_\text{ref}$}{s,\pc}{H}{}}

\inference
[NM]
{\dec(s,\pc) = (\nm, x) & \exec(s, \pc, (\nm, x)) = (s', \pc')}
{\ioe{P$_\text{ref}$}{s,\pc}{s', \pc'}}

\inference
[Ld]
{\dec(s,\pc) = (\ld, x, a) & \exec(s, \pc, (\ld, x, v)) = (s', \pc')}
{\io{P$_\text{ref}$}{s, \pc}
{s', \pc'}{(\ld\req, a), (\ld\resp, v), \Real}}

\inference
[St]
{\dec(s,\pc) = (\st, a, v) & \exec(s, \pc, (\st)) = (s', \pc')}
{\io{P$_\text{ref}$}{s, \pc}
{s', \pc'}{(\st, a, v)}}
\end{boxedminipage}

\caption{LTS for a single reference processor (P$_{\text{ref}}$)}
\label{Pref$}
\end{figure}


Figure \ref{Pref$} shows the LTS for a single processor that will be used as a
reference processor.  The definition is parametrized over details of an
\emph{instruction set architecture} (ISA).  In particular, the ISA gives us
some domains of architectural states $s$ (e.g., register files) and of program
counters $\pc$.  A function $\dec(s, \pc)$ figures out which instruction $\pc$
references in the current state, returning the instruction's ``decoded'' form.
A companion function $\exec(s, \pc, \mathit{dec})$ actually executes the
instruction, returning the new state $s'$ and the next program counter $\pc'$.

The legal instruction forms, which are outputs of $\dec$, are $(\nm, x)$, for
an operation not accessing memory; $(\ld, x, a)$, for a memory load from
address $a$; $(\st, a, v)$, for a memory store of value $v$ to address $a$; and
$H$, for a ``halt'' instruction that moves the LTS to state $H$. The parameter
$x$ above represents the rest of the instruction, including the opcode,
registers, constants, \etc{}

The label generated for a load contains the request sent to the
memory system, and the response obtained, as well as a label suffix $\Real$
which denotes that the load is non-speculative (We will later discuss the
alternative, $\Spec$ which arises in the load labels for our optimized
speculative processor).

The legal inputs to $\exec$ encode both a decoded instruction and any relevant
responses from the memory system.  These inputs are $(\nm, x)$ and $\st$, which
need no extra input from the memory; and $(\ld, x, v)$, where $v$ gives the
contents of the requested memory cell.

We can define sequential consistency using the simple memory from Figure
\ref{M_m} as follows, for $n$ processors:
\begin{defn}
$SC \triangleq P^n_\text{ref} + M_m$
\label{sc}
\end{defn}

We define the initial state of SC as $(\theta_0, m_0)$, where $m_0$ is
some initial memory fixed throughout our development, mapping every
address to value $v_0$; and $\theta_0$ maps every processor ID to
$(\pc_0, s_0)$, using architecture-specific default values $s_0$ and
$\pc_0$.

This LTS encodes Lamport's notion of SC, where processors take turns executing
nondeterministically in a simple interleaving.  We need know very few details
of the ISA to define the SC model.  Our final, optimized system is parametrized
over an ISA in the same way.  In the course of the rest of this paper, we will
define an optimized system $O$ and prove $O \sqsubseteq \text{SC}$.

Note that, in this setting, an operational specification like the LTS
for SC is precisely the right characterization of \textbf{full functional
  correctness} for a hardware design, much as a
precondition-postcondition pair does that job in a partial-correctness
Hoare logic.  Our SC LTS fully constrains observable behavior of a
system to remain consistent with simple interleaving.  Similar
operational models are possible as top-level specifications for
systems following weaker memory models, by giving the LTS for the \emph{Local
Buffer} component and composing the three components simultaneously.


\section{Speculative Out-Of-Order Processor}\label{sec:ooo}

The previous section divided a multiprocessor system into two rather
simple components, processors and memories.  We are now ready to begin
developing more complex, optimized versions of each piece, starting
with processors.

We implement a \emph{speculative} processor, which may create many
simultaneous outstanding requests to the memory, as an optimization to
increase parallelism.  Our processor proof is in some sense generic
over correct speculation strategies.  We parametrize over two key
components of a processor design: a \emph{branch predictor}, which
makes guesses about processor-local control flow in advance of
%receiving enough memory responses to resolve
resolving conditional jumps; and a
\emph{reorder buffer}, which decides what speculative instructions (like memory loads)
are worth issuing at which moments (in effect \emph{reordering} later
instructions to happen before earlier instructions have finished).

The branch predictor is the simpler of the two components to model.
We indicate branch-predictor state with metavariable $\ppc$.
The operations on such state are $\curPpc(\ppc)$, to extract the
current program-counter prediction; $\nextPpc(\ppc)$, to ask the
predictor to step forward by one instruction; and $\setBp(\ppc, \pc)$,
to reset prediction to begin at a known-accurate position $\pc$.  It
turns out that we need to impose no explicit correctness criterion on
branch predictors.  The processor uses predictions only as hints, and
it always resets the predictor using $\setBp$ after detecting an
inaccurate hint.

The interface and formal contract of a reorder buffer are more
involved.  We write $\rob$ as a metavariable for reorder buffer
state.  The associated operations are:
\begin{itemize}
\item $\phi$, the state of an empty buffer.
\item $\add(\pc, \rob)$, which appends the program instruction at location $\pc$ to the list of instructions that the buffer is allowed to consider executing.
\item $\get\spc\ld(\rob)$, which returns a set of speculative loads that can be issued to the memory system
\item $\compute(\rob)$, which models a
step of computation inside the buffer. For instance, it invokes the $\dec$
and $\exec$ functions (as defined for SC) internally to obtain the next program
counter, state, \etc{} (only the speculative states are updated, not the actual states).
\item $\upd\ld(\rob,t,v)$, which informs the buffer that the memory
has returned result value $v$ for the speculative load with tag $t$.
\item $\commit(\rob)$, which returns the next instruction in serial
program order, if we have accumulated enough memory responses to execute it
accurately, or returns
$\epsilon$ otherwise.  When $\commit$ returns an instruction, it also
returns the associated program counter plus the next program counter
we would advance to afterward.
 Furthermore, the instruction is
extended with any relevant response from memory (used only for load
instructions, obtained through $\upd\ld$) and with the new architectural state (\eg{} register
file) after execution.
\item $\retire(\rob)$, which informs the buffer that its $\commit$
instruction was executed successfully, so it is time to move on to the
next instruction.
\end{itemize}

We postpone characterization of correct reorder buffers until after we
present the LTS for speculating processors.

\begin{figure*}[t]
\small
\centering
\begin{boxedminipage}[c]{.95\textwidth}
\inference[Fetch]
{}
{\ioe{P$_\text{so}$}{s,\pc,\rob,\ppc}{s,\pc,\add(\curPpc(\ppc),\rob),\nextPpc(\ppc)}}

\inference[Compute]
{}
{\ioe{\text{P$_\text{so}$}}{s,\pc,\rob,\ppc}{s,\pc,\compute(\rob),\ppc}}

\inference[SpecLd]
{(\spc\ld,t,a) \in \get\spc\ld(\rob)}
{\io{\text{P$_\text{so}$}}{s,\pc,\rob,\ppc}{s,\pc,\upd\ld(\rob,t,v),\ppc}
{(\ld\req, a), (\ld\resp, v), \Spec}}

\inference[Abort]
{\commit(\rob) = (\pc',\_,\_) & \pc' \neq \pc}
{\ioe{\text{P$_\text{so}$}}{s,\pc,\rob,\ppc}{s,\pc,\phi,\setBp(\ppc,\pc)}}

\inference[Halt]
{\commit(\rob) = H}
{\ioe{\text{P$_\text{so}$}}{s,\pc,\rob,\ppc}{H}}

\inference[Nm]
{\commit(\rob) = (\pc,\pc',(\nm,s'))}
{\ioe{\text{P$_\text{so}$}}{s,\pc,\rob,\ppc}{s',\pc',\retire(\rob),\ppc}}

\inference[St]
{\commit(\rob) = (\pc,\pc',(\st,a,v,s'))}
{\io{\text{P$_\text{so}$}}{s',\pc',\retire(\rob),\ppc}{s,\pc,\rob,\ppc}
{\st, a, v}}

\inference[LdGood]
{\commit(\rob) = (\pc,\pc',(\ld,x,a,v,s'))}
{\io{\text{P$_\text{so}$}}{s,\pc,\rob,\ppc}{s',\pc',\retire(\rob),\ppc}{(\ld\req, a), (\ld\resp, v), \Real}}

\inference[LdBad]
{\commit(\rob) = (\pc,\pc',(\ld,x,a,v',s')) & v' \neq v & \exec(s,\pc,(\ld,x,v)) = (s'', \pc'')}
{\io{\text{P$_\text{so}$}}{s,\pc,\rob,\ppc}{s'',\pc'',\phi,\setBp(\ppc,\pc)}{(\ld\req, v), (\ld\resp,v), \Real}}
\end{boxedminipage}

\caption{Speculating, out-of-order issue processor}
\label{Ooo}
\end{figure*}

Figure~\ref{Ooo} defines that LTS, P$_{\texttt{so}}$. This processor may issue
arbitrary speculative loads (with label suffix $\Spec$), but it only ever
\emph{commits} the next instruction in serial program order. These loads are
issued non-deterministically from any load instruction after its address is
computed.  This non-deterministic issue is going to be necessary when the
memory LTS $M_m$ is replaced by caches.  To maintain SC, every speculative load
must have a matching verification load later on (with label suffix $\Real$),
and we maintain the illusion that the program only depends on the results of
verification loads, which, along with stores, \emph{must be issued in serial
program order}.

When committing a
previously issued speculative load instruction, the associated speculative
memory load response is verified against the new commit load response. If the
resulting values do not match, the processor terminates all past uncommitted
speculation, by emptying the reorder buffer and resetting the
next predicted program counter in the branch predictor to the correct next value. This step mimics
the mechanism implemented in most modern-day processors, operating on the
conservative assumption that the processor may have performed arbitrary
incorrect speculation after receiving the inaccurate response from
memory.  (Note that here ``inaccurate'' does not mean ``wrong at
the time the response was received,'' but rather ``wrong at commit
time'' or ``correct before, but not anymore.'')

A full processor state is $(s, \pc, \rob, \ppc)$, comprising
architectural state, the program counter, and the reorder buffer and branch
predictor states. Its initial state is given by $(s_0, \pc_0, \phi, \ppc_0)$.
The interface of this processor with memory is identical
to that of the reference processor, but now we can exercise a memory
system's transitions even for speculative loads.

\begin{figure*}[t]
\small
\begin{inv}
If $P_\text{so}$ reaches a state $(s, \pc, \rob, \ppc)$, \ie{}

$\io{P$_\text{so}^{*}$}{s_0,\pc_0,\phi,\ppc_0}{s,\pc,\rob,\ppc}{\eta} \Rightarrow$
\begin{displaymath}
\left\{
\begin{array}{ll}
\commit(\rob) = (\pc, \pc', (\nm, s')) &\Rightarrow
\exists x. \; \dec(s,\pc) = (\nm, x) \wedge \exec(s,\pc, (\nm,x)) =
(s', \pc')\\
\commit(\rob) = (\pc, \pc', (\ld, x, a, v, s')) & \Rightarrow
\dec(s,\pc) = (\ld, x,a) \wedge \exec(s,\pc, (\ld,x,v)) = (s',\pc')\\
\commit(\rob) = (\pc, \pc', (\st, a, v, s')) &\Rightarrow
\dec(s,\pc) = (\st, a,v) \wedge \exec(s,\pc, (\st)) =
(s',\pc')\\
\commit(\rob) = H & \Rightarrow
\dec(s, \pc) = H\\
\end{array}
\right.\end{displaymath}
\label{rob}
\end{inv}
\vspace{-.5cm}
\caption{Correctness of reorder buffer}
\label{robfig}
\end{figure*}

Finally, we impose a general correctness condition on reorder
buffers.  They must maintain Invariant~\ref{rob} in Figure~\ref{robfig}.
Intuitively, whenever the buffer claims in a $\commit$
output that a particular instruction is next to execute, causing
certain state changes, that instruction must really be next in line according
to how the program runs in the SC system, and running it must really cause
those state changes.

When this condition holds, we may conclude the correctness theorem for
out-of-order processors.  We use a trace-transformation function
$\noSpec$ that drops all speculative-load requests and responses.
See Definition~\ref{refines} for a review of how we use such
functions in framing trace refinement.  Intuitively, we prove that any
behavior by the speculating processor can be matched by the simple
processor, with speculative messages erased.

\begin{theorem}
\label{ocorrect}
$\text{P$_\text{so}$} \sqsubseteq_\noSpec$ P$_\text{ref}$
\end{theorem}
\begin{proof}
By induction on P$_\text{so}$ traces, using a proper abstraction
function (which drops the speculative messages and the $\rob$ and $\ppc$ states) to relate the two systems.  Invariant~\ref{rob} is crucial to
relate the reorder buffer's behavior with the simple in-order
execution of P$_\text{ref}$.
\end{proof}

\begin{corollary}
\label{ges}
$\text{P$^n_\text{so}$} \sqsubseteq_{\noSpec^n}$ P$_\text{ref}^n$
\end{corollary}
\begin{proof}
Direct consequence of Theorems~\ref{ocorrect} and~\ref{liftn} (where the
latter is one of our preliminary results about $n$-repetition).
\end{proof}
