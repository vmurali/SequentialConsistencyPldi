\section{Introduction}
\label{sec:Introduction}

A modern high-performance, cache-coherent, distributed-memory hardware system is
inherently complex. Such a system implements memory models like sequential
consistency (SC)~\cite{lamport1979make}, Total Store Order (TSO), and
Relaxed Memory Order (RMO)~\cite{weaver1994sparc}, each
of which characterizes how loads and stores may be issued by a processor and
how the shared-memory component responds to these requests.  These systems by
their nature are highly concurrent and nondeterministic.

The goal of this work is to provide a framework for full verification of such
complex hardware systems. Hardware designers already recognize the challenge of
implementing these systems, to such an extent that they have already become some
of the most serious real-world adopters of formal methods.  Hardware
verification is dominated by model-checking, where abstraction techniques are
used to reduce designs to finite state spaces, which can be explored
exhaustively.  There are limits to the construction of sound abstractions, so
verifications of, e.g., cache-coherence protocols have almost always treated
systems with concrete topologies, involving particular finite numbers of caches
and processors. For instance, explicit-state model checking tools like
Murphi~\cite{murphi} or TLC~\cite{tlc} are only able to handle small and
simplified systems, \eg{} a single-level cache hierarchy with three CPUs, two
addresses, and one bit of data. Symbolic model checking techniques have fared
slightly better: McMillan \etal{} have verified a 2-level MSI protocol based on
the Gigamax distributed multiprocessor using SMV~\cite{gigamax}. Optimizations
on these techniques (\eg{} partial order reduction~\cite{part}, symmetry
reduction~\cite{sym1, sym2}, compositional reasoning~\cite{comp}, \etc{}) scale
the approach further, but they are still unable to handle realistic
hierarchical protocols. We might hope to achieve higher assurance and
understanding of design ideas by verifying \emph{infinite families} of hardware
designs, which resist reduction to finite-state models.  A potential related
benefit is reducing the computational resources required to formally verify a
new design in the family, applying a more general theorem without much new
analysis.

Modularity has long been understood as a key property for effective design and
verification of such complex systems. For the designer, it allows a separation
of concerns, increasing robustness by allowing the behavior encapsulated by a
modular boundary to be realized by multiple implementations, any of which may be
dropped into the system safely. It also allows a greater measure of
parallelization of human design effort, improving development time. Similarly, verification is
simplified, as modular interface agreements provide a natural lemma structure.
This leads us towards a decomposition of the whole system verification
task into lemmas of subsystems, which can be composed in a black-box manner to
produce full-system theorems.

In this paper, we introduce \textbf{a new approach to modular,
  deductive verification of hardware designs} at the highest level of
rigor, backed by proofs using the Coq proof assistant.  As a
challenging case study, we focus on \textbf{verifying that a
  particular infinite family of multicore, shared-memory systems
  implements sequential consistency}.  The proof is parametrized over
an unknown number of processors connected to an arbitrary memory
hierarchy with an unknown number of caches in an unknown number of
layers (e.g., L1, L2).  In our design, processors and memory systems
both independently employ intricate optimizations that exploit
opportunities for parallelism.  We are able to prove that each of
those two main components still implements strong enough semantics to
support SC, and then we compose those theorems into a result for the
full system.  Either component may be optimized further without
requiring any changes to the implementation, specification, or proof
of the other.

To structure our hardware verification framework, we import
\emph{labeled transition systems (LTSes)}, a well-studied approach from the
world of concurrent software modeling.  As our unified notion of
specification and proof, we adopt \emph{trace refinement}, which
captures when one concurrent system can only produce observable IO
behaviors that another system could also produce.  Each of our
realistic hardware components is associated with a simpler ``reference
implementation'' that serves as a specification, and we prove that
each realistic component refines its spec.  Thereafter, it is sound to
use the simpler spec component in reasoning about the behavior of the
system.

Each key component of our case-study system has a simple spec: the
ideal memory responds instantly and atomically to each load or store
request, and the ideal processor executes instructions in order with
no speculation.  The optimized implementation of each is much more
complicated: the cache-hierarchy memory system potentially allows
every cache to be busy simultaneously computing and sending messages,
and the optimized speculative processor sends multiple load requests
to the memory at once, without waiting for responses first.  The meat
of the verification is in showing that each optimized component
refines its simpler counterpart.  After each such proof, we may in
effect substitute the optimized version for the simple version in a
black-box way within a design, with a full guarantee of soundness.

LTSes as hardware descriptions are an established
idea~\cite{HoeArvind:TRSSynthesis1, Hoe:TCAD}, and there are
compilers that convert LTSes into efficient hardware.  Our work is
based on the Bluespec language~\cite{BSV:LangRef, Bluespec:TFRG},
which served as the model for the formalism of this paper.  Bluespec
models hardware components as atomic rules of a transition system over
state elements, and there is a commercial compiler from such code into
register transfer-level circuits (i.e., Verilog code) with competitive
performance.  Our cache-coherent memory system is directly
transliterated from a Bluespec
implementation~\cite{DNA:CoherenceImplementation} used to
implement a multi-processor PowerPC system~\cite{Khan:PowerPc}. The
hardware synthesized from that implementation is rather efficient:
an 8-core system can run 55 million instructions per second on
the BEE FPGA board\cite{}, with a 2-level cache hierarchy.

The main contributions of this paper are:

\begin{itemize} 
\item Identification of an interface between the processor and the memory
components of a multi-processor system that enables modular implementation and
verification.

\item A general \emph{modular verification methodology} applicable to hardware
  algorithms, based on \emph{labeled transition systems}.  We are able to verify
  the several components of Figure~\ref{zoom} \adam{OK to have a
    forward reference to this figure?} against general formal interfaces,
  enabling mixing and matching of different realizations of each interface,
  without doing any new proofs that peek beneath abstraction boundaries.

\item The \emph{first machine-checked proof} of an implementation of sequential
  consistency involving speculative processors and cache-coherent distributed
  memory.

\item The \emph{first machine-checked proof} of an invalidation-based
  cache-coherence protocol for distributed memories comprised of arbitrary
  cache hierarchies. We believe such a memory is also the correct abstraction for
  implementing relaxed memory models.
\end{itemize}

\paragraph{Paper Organization:} In Section~\ref{sec:lts} we introduce our
flavor of the labeled transition systems formalism, including a definition of
trace refinement. In Section~\ref{sec:store-atomicity}, we show a generic
decomposition of any multi-processor system independently of the memory model
that it implements, and we discuss the store-atomicity property of the memory
sub-component. In Section~\ref{sec:sc} we
give a simple formal model of sequential consistency.  The
following sections refine the two main subcomponents from Figure~\ref{zoom}.
Section~\ref{sec:ooo} discusses definition and verification of a speculative
processor model, and Sections~\ref{sec:cc} and~\ref{sec:ccproof} respectively
define and prove our hierarchical cache-coherence protocol.  Finally, in
Section~\ref{sec:finalresult} we show the whole-system modular proof of our
complex system, finishing with discussion of related work and other conclusions in
Sections~\ref{relatedWork} and \ref{sec:conclusion}.

%% The goal of this work is to introduce an interface enabling true modular
%% implementation and verification of shared memory systems. That is, we should be
%% able to describe a modular decomposition where we may correct replace either
%% the processor nodes, the memory with locally verified alternatives without
%% changing the correctness of our whole system proof of correctness or
%% restricting our ability to describe shared memory systems. As part of this we
%% will describe the processor interface in a way which allows the full range of speculation 

%% We do so by showing the construction and initial refinements of a sequentially
%% consistent shared memory system. We formally defining a minimal cache coherent
%% speculation-friendly memory interface (\emph{Store Atomicity}), and use this as
%% the interface for our memory subsystem. Using this we specify the notion of
%% correct for a processor and prove that our system is correct assuming the
%% correct implementation of these two objects.  We then construct both a naive
%% reference processor and memory subsystem and provide proofs of correctness in
%% Coq. We then refine both components into more realistic implementations
%% including taking our memory subsystem to hardware-level description of a
%% arbitrary-level hierarchical cache coherence implementation and verify these
%% module. We show that the cross products of possible implementations are correct
%% and their proof automatically from the modularly.


%The traditional notion of modularity in hardware is closely tied to
%\emph{Finite State Machines} (FSMs), more specificially synchronous FSMs. The biggest
%problem with using FSM abstraction is that it restricts significantly the kinds of
%refinements that can be performed on a module. For example, if we change the
%timing of an adder so that it takes 2 clock cycles instead of 1, the whole
%system is likely to break. Even though big modules in hardware are often
%designed with asynchronous handshaking protocols, current
%hardware verification methodologies do not deal with timing refinements.
%In this paper, we use the formalism of \emph{labeled transition systems} that
%is well-known in process-calculus circles.  It turns out to be exactly the
%right way of capturing contracts on interaction between hardware components,
%enabling both modular refinement and verification.

%\begin{figure}
%\centering
%\includegraphics[scale=.43]{zoom}
%\caption{Overall system}
%\label{zoom}
%\end{figure}
%
%Figure \ref{zoom} sketches the sort of hardware system that we have verified.
%We show the top-level system design at top right, and we zoom in on boxes standing for the
%memory system and a single processor.
%
%Memory is composed of a \emph{hierarchy of caches}, where each cache node (labeled
%like ``L1,'' ``L2,'' etc.) communicates only with its neighbors in the graph.
%Each processor has a dedicated L1 cache, from which we do our best to satisfy
%all memory requests, to avoid the latency of a round-trip with main memory.
%With an L1 cache miss, we may still find a hit in a parent cache below the
%level of main memory, realizing a smaller but still significant speedup.
%We have verified a \emph{directory-based} protocol for coordinating an arbitrary
%tree of caches, where each node stores a conservative approximation of its
%children's states.
%
%A processor is decomposed into several components.  We have the normal
%architectural state, such as values of registers.  Our proofs are generic over
%\emph{a family of instruction set architectures}, with parameters for opcode sets and
%functions for executing opcodes and decoding them from memory.  Other key
%components are a \emph{branch predictor}, which guesses at the control-flow path
%that a processor will follow, to facilitate speculation; and a
%\emph{reorder buffer (ROB)}, which decides which instructions along that path to
%try executing ahead of schedule.  Our proofs apply to an arbitrary branch predictor,
%and they work for any reorder buffer satisfying a simple semantic condition.
%
%Figure~\ref{proofs} gives the overall proof structure that we employ to verify
%the system of Figure~\ref{zoom}. As we will explain later, $P_\text{so}$
%represents the speculative out-of-order processor, $M_c$ the cache memory,
%$M_m$ the simple memory, $P_\text{ref}$ a simple decoupled processor, and
%SC the simple reference model having multiple processors executing each
%instruction atomically thus implementing sequential consistency.  Notation
%$A^n$ is for $n$ copies of system $A$ running in parallel.
%$\sqsubseteq$ is a refinement operator, capturing a suitable notion of when
%one system implements another.  We go into more detail on the symbols from
%the figure in the rest of the paper.
%
%Our current framework establishes theorems of the form ``if system $A$ has a run
%with some particular observable behavior, then system $B$ also has a run with
%the same behavior.''  In this sense, we say that $A$ correctly implements $B$.
%Other important properties, such as \emph{deadlock freedom} for $A$ (which
%might get stuck without producing any useful behavior), we leave for future
%work.
%
%\begin{figure}
%\includegraphics[scale=.45]{proofs}
%\caption{Overall proof structure}
%\label{proofs}
%\end{figure}


%% Modularity has long been understood as a key property for effective design and
%% verification of complex systems, \eg\ distributed memory systems. For the
%% designer, it allows a separation of concerns increasing robustness by allowing
%% the behavior encapsulated by a modular boundary to be realized by multiple
%% implementation any of which may be safely dropped into the system. It also
%% allows a greater measure of parallelization of design improving development
%% time. Similarly verification is simplified as modular interface agreements
%% provide a set of natural lemmas to verify. This leads us towards a
%% decomposition of the whole system verification task into smaller more
%% manageable chunks.

%% Given its complexity, it is a practical necessity for distributed shared memory
%% systems to be built a strong idea of modularity. The standard modularization
%% (see Figure~\ref{fig:highleveldsm}) separates the ISA-level processors nodes
%% which are responsible for executing threads of computation, and a unified
%% shared memory subsystem which is responsible for ensuring that memory requests
%% it receives are handled correctly, \ie{} in line with the memory model which
%% dictates which store values a load may observe in the system. These processors
%% may then be refined for performance, while the monolithic memory module is
%% improved by introducing layers of caching and a coherence protocol to guarantee
%% memory consistency.

%% This high-level approaches works reasonably well for implementation; separate
%% design groups tackle can each task in isolation. However, when we consider
%% verification, the modularity enforced during the implementation has little
%% benefit and in practice we must resort to full system verification. The core of
%% this problem lies in the fact that the simple high-level decomposition of
%% memory and processors does not fully take into account the space of speculation
%% techniques used in the processor.

%% All practical processor instruction set architectures (ISAs) are fairly
%% sequential; the result of executing an instruction may affect which next
%% instruction is executed. As a result for performance some sort of prediction is
%% used to allow instructions to be speculatively started and executed before the
%% previous instructions have completed and the values need to execute are known.
%% Additional bookkeeping is used to determine when our guess is wrong and
%% false-path instructions are undone leaving no affect on the architectural
%% state. As precise exception handling is a requirement these speculative
%% instructions are committed, \ie{} expose their effects of execution to the
%% whole system, in sequence.

%% Though most forms of speculation (\eg{} branch prediction, instruction prefetch
%% from pipelining, value prediction) are often considered as being orthogonal to
%% memory issues, speculation changes the sequence of memory requests the
%% processor sends to the memory. This may be merely handling requests out of
%% order, \eg{} issuing a load request before a logically earlier one has issued
%% (a common occurrence due to data dependencies in out-of-order processors). We
%% may also allow incorrect memory operations to be issued, \eg{} executing a load
%% from a false path. It is worth noting that while such load operations are
%% possible to speculate, we cannot speculatively issues store operations to the
%% memory as they become exposed to the whole system and speculation mechanisms
%% are isolated to a single processor. As a consequence of this, our memory does
%% not know what order requests it is sent should happen. Thus the memory in
%% isolation cannot directly realize the memory model and in actually implements a
%% much weaker model leaving the full enforcement of the memory model to the
%% processor which can enforce this by restricting when it issues requests. 

%% The goal of this work is to introduce an interface enabling true modular
%% implementation and verification of shared memory systems. That is, we should be
%% able to describe a modular decomposition where we may correct replace either
%% the processor nodes, the memory with locally verified alternatives without
%% changing the correctness of our whole system proof of correctness or
%% restricting our ability to describe shared memory systems. As part of this we
%% will describe the processor interface in a way which allows the full range of speculation 

%% We do so by showing the construction and initial refinements of a sequentially
%% consistent shared memory system. We formally defining a minimal cache coherent
%% speculation-friendly memory interface (\emph{Store Atomicity}), and use this as
%% the interface for our memory subsystem. Using this we specify the notion of
%% correct for a processor and prove that our system is correct assuming the
%% correct implementation of these two objects.  We then construct both a naive
%% reference processor and memory subsystem and provide proofs of correctness in
%% Coq. We then refine both components into more realistic implementations
%% including taking our memory subsystem to hardware-level description of a
%% arbitrary-level hierarchical cache coherence implementation and verify these
%% module. We show that the cross products of possible implementations are correct
%% and their proof automatically from the modularly.

%%  LocalWords:  Modularity modularity
