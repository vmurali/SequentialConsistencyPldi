\section{Conclusions and Future Work}\label{sec:conclusion}

In this paper we developed a modular proof structure for distributed
shared-memory hardware systems, corresponding naturally to the modularization seen in
hardware implementation. This makes it feasible for verification of the
system to be carried out throughout the implementation process without
disrupting the architect's design process.

While we provide a clean interface for an SC system, we are working on
encompassing relaxed memory models commonly used in a modern processors.

Another direction that bears investigation is extending our results to
cover finer-grained architectural details.
However, the behaviors of microarchitectural structures,
\ie{} register renaming structures \etc{}, vary significantly across
implementations and can be quite complex. It is not clear if a
practical modularization can be found that strikes a balance between flexibility and
simplicity.

Finally, to make full use of our results for implementations, hardware
descriptions must be translated to and from our labeled transition system
representation. As we implied in the discussion of our cache-coherence system,
Bluespec SystemVerilog's rule-based representation is very close to LTSes and
in some cases can be transliterated directly.  There has been work to generate
circuits from Bluespec-like descriptions in Coq \cite{Braibant2013Fesi} and we
are pursuing further automatization.
