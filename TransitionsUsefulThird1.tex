\section{Cache-Based Memory System}\label{sec:cc}

We now turn our attention toward a more efficient implementation of
memory.  It is physically infeasible to connect many processors
directly to a single memory with low enough latency to handle most
memory requests quickly enough.  Instead, each processor will have its
own \emph{L1 cache} of recently used memory cells.  Whenever possible, we
want to service memory requests from processors' own caches, only
contacting memory for addresses not found in the caches.  To further
avoid contacting main memory, we create additional higher-level
caches, connecting each L1 cache to an L2 cache, potentially
connecting L2 caches to L3, etc., bottoming out in main memory at the
root of our tree. The zoomed-in portion of the memory in Figure~\ref{zoom}
diagrams the basic layout.

Now we have concurrent interaction of many processors with many caches
with main memory, and the relationship with the original $M_m$ system
is far from direct.  However, this intricate concurrent execution is
crucial to hiding the latency of main-memory access.
Figure~\ref{cache} formalizes as an LTS $M_c$ the algorithm we implemented,
for providing the memory abstraction on top of a cache hierarchy.
We have what is called an \emph{invalidating directory-based hierarchical
cache-coherence protocol}.  It is a transliteration of a realistic
algorithm implemented in the Bluespec hardware description
language~\cite{Bluespec:TFRG}. 

We now try to explain aspects of this transition system at an
intuitive level.  Thanks to the overall modular proof structure in
this paper, it is not essential to understand the details of the cache
system, to understand any other part of our development.  Therefore,
some readers may want to skip this explanation upon first reading,
skipping to the next section for a key theorem relating $M_c$ and $M_m$.

We describe a state of the system using 8 fields: $\dt$, $\ch$, $\s$, $\dst$,
$\wt$, $\dwt$, $\inp$, $\outp$. The $\inp$ and $\outp$ sets are the interfaces
to the processors and are exactly the same as in the simple memory LTS
(Figure~\ref{$M_m$}). The rest are summarized in Figure~\ref{fields}.

%We describe a state of the system using 6 fields: $\dt$, $\ch$, $\s$, $\dst$,
%$\wt$, $\dwt$ which are summarized in Figure~\ref{fields}.
We use $\parent(c,p)$ to denote that $p$ is the parent of $c$ and $\leaf(c)$ to 
denote that $c$ is a leaf or L1 cache.

\begin{figure}[h]
\centering
\begin{tabular}{|l|p{6cm}|}
\hline
$\dt(c,a)$ & Data in cache $c$ for address $a$\\
$\s(c,a)$ & Coherence state of cache $c$ for address $a$\\
$\dst(p,c,a)$ & Directory state of cache $p$ for child $c$ and address $a$\\
$\wt(c,a)$ & State that cache $c$ is waiting to upgrade to, for address $a$, or $\epsilon$\\ %Gives cache $c$'s desired coherence state to upgrade to (for address $a$), or $\epsilon$ if none\\
$\dwt(p,c,a)$ & State that cache $p$ is waiting its child $c$ to downgrade to, for address $a$, or $\epsilon$\\
%$\dwt(p,c,a)$ & Gives cache $p$'s desired coherence state for its child $c$ to downgrade to (for address $a$), or $\epsilon$ if none\\
$\ch$ & Channels between nodes in the system\\
\hline
\end{tabular}
\caption{Fields in a cache-based memory system}
\label{fields}
\end{figure}

A coherence state is $M, S$ or $I$, broadly representing permissions to modify,
read, or do nothing with an address, respectively, the decreasing permissions
denoted by $M > S > I$. This interpretation of permissions is accurate at the
leaf caches connected to the processors, but the
interpretation is subtly different at the higher level caches.

$\wt$ denotes what permission the cache is waiting for,
if any.  That is, a cache has decided to \emph{upgrade} its
coherence state for some address to a more permissive value, but
it is waiting for acknowledgment from its parent before
finalizing the upgrade.

$\dst$ represents the parent's notion of the
coherence state of the child for that address. We later prove that this notion
is always conservative, \ie{} if the parent assumes that a child does not have
a particular permission, then it is guaranteed in this system that the child
will not have that permission.  $\dwt$
denotes whether the parent is waiting for any downgrade response from the
corresponding child, and if so, the coherence state that the child must
downgrade to as well.

All communication channels in the system are represented by field $\ch$.  The
channels are asymmetric as follows. Channels going from parent to children
(indexed by the names of the parent and child) maintain FIFO order of messages
between the same parent-child pair. They carry both request messages (asking
the child to downgrade) and response messages (acknowledging that the child can
upgrade). Channels going from children to their parents are further split into
request and response channels, thus indexed by tuples (child name, parent name,
request or response).  These channels need not maintain the FIFO order for
messages between the same child-parent pair (notice the use of $\coloncolon$ to
add and remove elements from the parent to child channel, as opposed to
$\uplus$ for the child to parent channel). The reason is that only one request
or one response is sent for a particular address between a child-parent pair --
this invariant can be proved for the LTS in  Figure \ref{cache}. The order
between a request and response message sent from the same child to its parent
is lost, but our protocol ensures that they are processed in order by the parent, by
comparing the previous coherence state of the incoming messages with the
parent's directory state (in rules ParentRecvRsp
and ParentRecvReq).

Here is an intuition on how the transitions work in the common case.  A cache
can decide, spontaneously, to upgrade its coherence state, in which case it
sends an upgrade request to its parent. The parent then makes a local decision
on whether to send a response to the requesting child or not, based on its
directory approximation and its own coherence state $\s$. If $\s$ is lower than
the requested upgrade, then it cannot handle the request, and instead must
decide to upgrade $\s$.  Once the parent's $\s$ is not lower than the requested
upgrade, it makes sure that the rest of its children are ``compatible'' with
the requested upgrade (given by $\compat$ definition below).  If not, the
parent must send requests to the incompatible children to downgrade. Finally, when
the $\s$'s upgrade and children's downgrade responses are all received, the
original request can be responded to. The transitions for sending requests are
allowed to happen spontaneously without any trigger. $P_\text{ref}$ issues speculative
loads only for addresses which are in \Sh{} or \Mo{} state in L1; otherwise L1
spontaneously requests an upgrade for that address.

\begin{defn}
$\compat(p, c, x, a) =$
%\vspace{.3cm}

\begin{math}
\left\{
\begin{array}{ll}
\forall c' \neq c. \; \dst(p,c',a) = I &: x = M\\
\forall c' \neq c. \; \dst(p,c',a) \le S &: x = S\\
\end{array}
\right.
\end{math}
\end{defn}

The reason for having separate request and response channels for the same
child-parent pair is to avoid deadlocks. Throughout the protocol, we maintain
the invariant that a downgrade request from the parent always gets priority,
\ie{} if a child is waiting for an upgrade response, it must still try to
satisfy a downgrade request from the parent. If both the parent and the child
are waiting for their respective responses, there will be a deadlock. This tie
cannot be broken arbitrarily: if the parent concedes and starts processing a
child's request, then it may have to send downgrade requests to other children
to make the states compatible, but the other children may have sent upgrade
requests and remain waiting, leading to a deadlock. The only way to break the
vicious waiting cycle is for the children to concede and respond when they get
downgrade requests. Of course, they have to downgrade the states of their own
children first, and this process goes on recursively until the downgrade
requests reach the leaf caches, which can respond immediately.

A complication arises because a cache can voluntarily decide to downgrade its
state.  This transition is used to model invalidation of cache lines to make
room for a different location (we permit a cache to downgrade from $M$ to $S$
instead of only to $I$).
As a result, the parent's $\dst$ and the
corresponding $\s$ of the child may go out of sync, leading to the parent
requesting a child to downgrade when it already has. To handle this situation,
the child has to drop the downgrade request when it has already downgraded to
the required state (Rule DropReq in Figure \ref{cache}), to avoid deadlocks
by not dequeuing the request.

\begin{figure*}
\centering
\small

\begin{boxedminipage}[c]{.95\textwidth}

\textbf{Processor/Memory Interface}

\inference
[Insert]
{}
{\iorn{$M_c$}{\dt, \ch, \s, \dst, \wt, \dwt, \inp, \outp}{\dt, \ch,
    \s, \dst, \wt, \dwt, \inp[i \coloneqq \{q\} \uplus \inp(i)],
    \outp}{i}{q}}
\vspace{.1in}
\inference
[Remove]
{\outp(i) = \mathit{rs} \uplus \{r\}}
{\ioln{$M_c$}{\dt, \ch, \s, \dst, \wt, \dwt, \inp, \outp}{\dt, \ch, \s, \dst, \wt, \dwt, \inp, \outp[i \coloneqq \mathit{rs}]}{i}{r}}
\vspace{.1in}
\inference[SpecLoad]
{\inp(c) = \{(\spc\ld, t, a)\} \uplus \rs & \s(c,a) \geq S}
{\ioxe{$M_c$}
{\dt, \ch, \s, \dst, \wt, \dwt, \inp, \outp}
{
\dt, \ch, \s, \dst, \wt, \dwt, \inp[c \coloneqq \rs],\\
\outp[c \coloneqq \outp(c) \uplus \{(\spc\ld, t, \dt(c, a))\}]
}}
\vspace{.1in}
\inference[Load]
{\inp(c) = \{(\ld, a)\} \uplus \rs & \s(c,a) \geq S}
{\ioxe{$M_c$}
{\dt, \ch, \s, \dst, \wt, \dwt, \inp, \outp}
{\dt, \ch, \s, \dst, \wt, \dwt, \inp[c \coloneqq \rs], \\
\outp[c \coloneqq \outp(c) \uplus \{(\ld, \dt(c, a))\}]}}
\vspace{.1in}
\inference[Store]
{\inp(c) = \{(\st, a, v)\} \uplus \rs & \s(c,a) \geq M}
{\ioxe{$M_c$}
{\dt, \ch, \s, \dst, \wt, \dwt, \inp, \outp}
{\dt[(c, a) \coloneqq v], \ch, \s, \dst, \wt, \dwt, \inp[c \coloneqq \rs],
  \\
\outp[c \coloneqq \outp(c) \uplus \{(\st)\}]}}

\textbf{Child Upgrade}

\inference[ChildSendReq]
{\parent(c,p) & \s(c,a) < x & \wt(c,a) = \epsilon}
{\ioxe{$M_c$}
{\dt, \ch, \s, \dst, \wt, \\ \dwt, \inp, \outp}
{\dt, \ch[ (c, p, \req) \coloneqq (a, \s(c, a), x) \uplus \ch(c,p,\req)], \s, \dst,\\
\wt[(c,a)\coloneqq x], \dwt, \inp, \outp}}
\vspace{.1in}
\inference[ParentRecvReq]
{\parent(c,p) & \ch(c, p, \req) = \{(a, y, x)\} \uplus \rs & \s(p, a) \geq x
\\ \compat(p, c, x, a) & \dwt(p,c,a) = \epsilon & \dst(p,c,a) \le y}
{\ioxe{$M_c$}
{\dt, \ch, \s, \dst,\\ \wt, \dwt, \inp, \outp}
{\dt, \ch[(c,p,\req) \coloneqq \rs][(p, c) \coloneqq (\resp, (a, \dst(p,c,a), x,\\ \;\;\;\;
\ite{\dst(p,c,a) = I}{\dt(p,a)}{\_})) \coloncolon \ch(p,c)], \\ \s, \dst[(p,c, a)\coloneqq x],
\wt, \dwt, \inp, \outp}}
\vspace{.1in}
\inference[ChildRecvRsp]
{\parent(c,p) & \ch(p, c) = \rs \coloncolon (\resp, (a, y, x, v))}
{\ioxe{$M_c$}
{\dt, \ch, \s, \dst, \wt, \\ \dwt, \inp, \outp}
{\dt[(c,a) \coloneqq \ite{y = I}{v}{\dt(c,a)}], \ch[(p,c) \coloneqq \rs], \s[(c,a) \coloneqq x], \dst,\\
\wt[(c,a)\coloneqq \ite{\wt(c,a) \le x}{\epsilon}{\wt(c,a)}], \dwt, \inp, \outp}}

\textbf{Parent Downgrade}

\inference[ParentSendReq]
{\parent(c,p) & \dst(p,c,a) > x & \dwt(p,c,a) = \epsilon}
{\ioxe{$M_c$}
{\dt, \ch, \s, \dst, \wt, \\ \dwt, \inp, \outp}
{\dt, \ch[(p, c)\coloneqq (\req, (a, \dst(p, c, a), x)) \coloncolon \ch(p,c)], \s, \dst,\\
\wt, \dwt[(p, c,a)\coloneqq x], \inp, \outp}}
\vspace{.1in}
\inference[ChildRecvReq]
{\parent(c,p) & \ch(p, c) = \rs \coloncolon (\req, (a, y, x)) & (\forall i. \; \parent(i, c) \Rightarrow \dst(c, i, a) \le x)
& \s(c,a) > x}
{\ioxe{$M_c$}
{\dt, \ch, \s, \dst,\\ \wt, \dwt, \inp, \outp}
{\dt, \ch[(p,c) \coloneqq \rs][(c, p,\resp) \coloneqq (a, \s(c,a), x,\\ \;\;\;\;
\ite{\dst(c,a) = M}{\dt(c,a)}{\_}) \uplus \ch(c,p,\resp)], \\ \s[(c,a)\coloneqq x], \dst,
\wt, \dwt, \inp, \outp}}
\vspace{.1in}
\inference[ParentRecvRsp]
{\parent(c,p) & \ch(c,p,\resp) = \{(a, y, x, v)\} \uplus \rs & \dst(p,c,a) = y}
{\ioxe{$M_c$}
{\dt, \ch, \s, \dst, \wt, \\ \dwt, \inp, \outp}
{\dt[(p,a) \coloneqq \ite{y = M}{v}{\dt(p,a)}], \ch[(c,p,\resp) \coloneqq \rs], \s, \dst[(p,c,a) \coloneqq x],\\
\wt, \dwt[(p,c,a)\coloneqq \ite{\dwt(p,c,a) \ge x}{\epsilon}{\dwt(p,c,a)}], \inp, \outp}}

\textbf{Voluntary downgrade for replacement}

\inference[VolResp]
{\parent(c,p) & (\forall i. \; \parent(i, c) \Rightarrow \dst(c, i, a) \le x)
& \s(c,a) > x}
{\ioxe{$M_c$}
{\dt, \ch, \s, \dst,\\ \wt, \dwt, \inp, \outp}
{\dt, \ch[(c, p,\resp) \coloneqq (a, \s(c,a), x,\\ \;\;\;\;
\ite{\s(c,a) = M}{\dt(c,a)}{\_}) \uplus \ch(c,p,\resp)], \\ \s[(c,a)\coloneqq x], \dst,
\wt, \dwt, \inp, \outp}}

\textbf{Dropping request because of voluntary downgrade}

\inference[DropReq]
{\parent(c,p) & \ch(p, c) = \rs \coloncolon (\req, (a, y, x)) & 
 \s(c,a) \le x}
{\ioxe{$M_c$}
{\dt, \ch, \s, \dst, \wt, \dwt, \inp, \outp}
{\dt, \ch[(p,c) \coloneqq \rs], \s, \dst,
\wt, \dwt, \inp, \outp}}
\end{boxedminipage}

\caption{LTS for cache-coherent shared-memory system}
\label{cache}
\end{figure*}


\subsection{Proving $M_c$ is Store Atomicity}\label{sec:ccproof}
\label{safety}

We must prove the following theorem, \ie{} the cache-based system is sound with
respect to the simple memory.
\begin{theorem}
\label{ccorrect}
$M_c \sqsubseteq M_m$
\end{theorem}

The key property we need for this proof is what we define as
``store atomicity'' below. Throughout this section, we say
\emph{time} to denote the number of transitions that occurred before reaching
the specified state.

\begin{defn}
Store atomicity is defined as the following property:
For a load request $\toM(t, \ld, a)$
from any processor, the response sent at time $T$,
$\toP(t, \ld, v)$ should be such that
\begin{enumerate}
\item $v = v_0$ (the initial value of any memory address) and no store
  request $\toM(\st, a, v')$ has been processed at any time $T'$ such
  that $T' < T$ or
\item There is a store request $\toM(\st, a, v)$ that was processed at time $T_q$ such that
$T_q < T$ and no other store
request $\toM(\st, a, v')$ was processed at any time $T'$ such that $T_q < T' < T$.
\end{enumerate}
\label{sa}
\end{defn}

We can easily show that any system obeying store atomicity is sound with
respect to $M_m$. Intuitively, the store atomicity property precisely describes
the property of a memory, \viz{} that if a store happens, then that location
gets updated such that any subsequent load gets back the updated value.

The proof that $M_c$ obeys store atomicity is involved enough that we will only
state the lemmas used to prove this. The proofs in Coq for each theorem and
lemma in this section can be found in our supplementary material.

We believe that other invalidation-based protocols with inclusive caches (\ie{}
those cache hierarchies where a parent cache contains all the locations that
any of its children contains) like MESI, MOSI, and MOESI also satisfy our
lemmas, making their verification easier by reusing most of our MSI proof; we
omit the discussion of extension of this proof to other protocols for space
reasons.

\begin{lemma}
\textit{latestValue}:
At any time $T$, a cache $c$ that is \clean{} (defined below) for an address
$a$ will have the most up-to-date value for that address, \ie{}
\begin{enumerate}
\item $\dt(c,a) = v_0$ and no store request $\toM(\st, a, v)$ has been processed at
any time $T'$ such that $T' < T$ or
\item There is a store request $\toM(\st, a, v)$ that was processed at time $T_q$ such that
$T_q < T \wedge \dt(c,a) = v$ and no other store
request $\toM(\st, a, v')$ was processed at any time $T'$ such that $T_q < T' < T$.
\end{enumerate}
\label{latestValue}
\end{lemma}

\vspace{-.15in}

\begin{defn}
\textit{clean}: A cache is said to be \textit{clean} for an address $a$ if and only if
the state of the cache for that address is at least \Sh{} and the directory 
state for any of its children is at most \Sh.
\label{clean}
\end{defn}

It is pretty intuitive to prove the store atomicity property given
Lemma~\ref{latestValue}. A leaf cache can respond to a request only when it
has the required permission, at which point it is guaranteed to have the latest
value, because it is \textit{clean} (as it has no children). To prove Lemma
\ref{latestValue}, it has to be decomposed futher into the following, each of
which holds at any time.

\begin{lemma}
\textit{nonAncestorCompatible}: For two distinct caches $c_1$ and $c_2$, neither of which is the
ancestor of the other in the cache-tree,
%such
%that neither is an ancestor of the other in the cache-tree hierarchy (where
%the ancestor relation is the reflexive-transitive closure of the parent relation),
coherence states $\s(c_1,a)$ and $\s(c_2,a)$ are
\textit{compatible} (defined below) for every address $a$.
\label{nonAncestorCompatible}
\end{lemma}
\begin{defn}
\textit{compatible}: Coherence states $x_1$ and $x_2$ are \textit{compatible}
iff whenever either $x_1$ or $x_2$ is \Mo{}, the other is \In.
\label{compatible}
\end{defn}

\begin{lemma}
\textit{noTransitWrite}: Whenever data for an address $a$ is in transit (\ie{}
$\forall T. \; T_s \le T \le T_r$ where $T_s$ is the time of sending the data and
$T_r$ the time of receiving the data), no cache can process a store request for
$a$, and the data must be sent from a \textit{clean} cache.
\label{noTransitWrite}
\end{lemma}

\begin{lemma}
\textit{conservative}: A cache's state for an address is never greater than
the corresponding state in its parent's directory.
\label{conservative}
\end{lemma}

\begin{lemma}
\textit{localCompatibility1}:
The state of a cache is never less than its directory entry for any of
its children.
\label{localCompatibility1}
\end{lemma}
\begin{lemma}
\textit{localCompatibility2}: Let $c_1$ and $c_2$ be distinct caches with the
same parent $p$. Then, $\s(c_1, a)$ and $\s(c_2, a)$ are
\textit{compatible}.
\label{localCompatibility2}
\end{lemma}

\section{The Final Result}\label{sec:finalresult}

With our two main results about optimized processors and memories, it
is now straightforward to complete the correctness proof of the
composed optimized system.

First, we need to know that, whenever the simple memory can generate
some trace of messages, it could also generate the same trace with all
speculative messages removed.  We need this property to justify the
introduction of speculation, during our final series of refinements
from the optimized system to SC.

\begin{theorem}
\label{mcorrect}
$M_m \sqsubseteq_{\noSpec^n} M_m$
\end{theorem}
\begin{prf}
By induction on traces, with an identity abstraction function.
\end{prf}

That theorem turns out to be the crucial ingredient to justify placing
a speculative processor in-context with simple memory.

\begin{theorem}
\label{ocorrect1}
$\text{P$^n_\text{so}$} + M_m \sqsubseteq \text{P$^n_\text{ref}$} + M_m$
\end{theorem}
\begin{prf}
Follows from Theorem \ref{liftplus} (one of our preliminary results
about $+$), Corollary \ref{ges}, and Theorem \ref{mcorrect}.
\end{prf}

The last theorem kept the memory the same while refining the
processor.  The next one does the opposite, switching out memory.

\begin{theorem}
\label{ocorrect2}
$\text{P$^n_\text{so}$} + M_c \sqsubseteq \text{P$^n_\text{so}$} + M_m$
\end{theorem}
\begin{prf}
Follows from Theorems \ref{ccorrect} and \ref{liftplus} plus reflexivity of $\sqsubseteq$ (Theorem \ref{transitive}).
\end{prf}

\begin{theorem}
\label{ofull}
$\text{P$^n_\text{so}$} + M_c \sqsubseteq SC$
\end{theorem}
\begin{prf}
We twice apply $\sqsubseteq$ transitivity (Theorem \ref{transitive}) to
%connect Theorems \ref{ocorrect2} and \ref{ocorrect1}, and use Definition \ref{sc} of $SC$
connect Theorems \ref{ocorrect2}, \ref{ocorrect1}, and \ref{scthm}.
\end{prf}
