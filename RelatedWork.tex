\vspace{-.4cm}
\section{Related Work}
\label{relatedWork}

Hardware verification is dominated by model checking; for instance, processor
verification ~\cite{burch1994automatic, mcmillan1998verification} and more
recently, Intel's execution cluster verification~\cite{kaivola2009replacing}.
Many abstraction techniques are used to reduce designs to finite state spaces,
which can be explored exhaustively. There are limits to the construction of
sound abstractions, so verifications of, e.g., cache-coherence protocols have
almost always treated systems with concrete topologies, involving particular
finite numbers of caches and processors. For instance, explicit-state model
checking tools like Murphi~\cite{murphi} or TLC~\cite{tlc} are only able to
handle small and simplified systems, \eg{} a single-level cache hierarchy with
three CPUs, two addresses, and one bit of data. Symbolic model-checking
techniques have fared better: McMillan \etal{} have verified a 2-level MSI
protocol based on the Gigamax distributed multiprocessor using
SMV~\cite{gigamax}. Optimizations on these techniques (\eg{} partial order
reduction~\cite{part}, symmetry reduction~\cite{sym1, sym2}, compositional
reasoning~\cite{comp, Mccomp, mcc}, \etc{}) scale the approach further, but
they are still unable to handle realistic hierarchical protocols. We might hope
to achieve higher assurance and understanding of design ideas by verifying
\emph{infinite families} of hardware designs, which resist reduction to
finite-state models.  A potential related benefit is reducing the computational
resources required to formally verify a new design in the family, applying a
more general theorem without much new analysis.

Theorem provers have also been used to verify microprocessors, \eg{} HOL has
been used to verify an academic microprocessor AVI-1 \cite{windley1995formal}.
Term-rewriting techniques have been used to specify microprocessors, with
the goal of making them more amenable to verification \cite{shen1999using}.

Cache-coherence proofs have also been done with mechanized theorem provers. The
correctness of the FLASH cache-coherence protocol~\cite{flash} for a
single-level hierarchy was proven with PVS~\cite{park} using a mapping, called
``aggregation of transactions,'' from multiple localized transactions to a
single global transaction, and finally using model checking to prove the
correctness of the global transactions model. However, they do not provide a
methodology to prove the aggregation of transactions for other protocols, which
is where, we believe, the difficulty lies; in contrast, our modular proof
relies on generic invariants, making it more adaptable to different protocols.
Cachet~\cite{StoyShenArvind:Proofs}, another single-level cache coherence
protocol, was verified to obey the CRF memory model~\cite{Shen:CRF}, and
parts of the proof were implemented using PVS.

% Shen \etal{} provided a paper-and-pencil proof showing
%Cachet does not violate the CRF memory model, while Stoy \etal{} implemented
%portions of the paper-and-pencil proof using PVS. However, neither proof
%establishes that the protocol is store atomic. Rather, they show that Cachet
%obeys the CRF memory model, which is not restricted to store-atomic behavior.

Before verifying that a hardware design implements an ISA-level semantics, it
is necessary to decide what the correct semantics is.  Much recent work has
studied widely deployed hardware platforms empirically to derive formal
semantics.  While our case study in this paper hews to the uncontroversial
sequential-consistency model, the empirical challenges arise in characterizing
weaker memory models.  Approaches have included building axiomatic
memory-model specifications based on empirically observed behavior on
certain ``litmus
tests''~\cite{Alglave11,Alglave:FMSD,alglave2012formal,Alglave:2010,Alglave:TACAS,alglave2014herding},
building models that include operational semantics~\cite{mador2012axiomatic,sarkar2011understanding,sarkar2012synchronising}, and relating the axiomatic specification of
TSO to operational semantics~\cite{x86tsocacm10}.

%There has been a lot of work in providing frameworks for axiomatic
%specifications of memory models \cite{alglave2012formal,alglave2014herding}
%and in providing axiomatic memory model specifications for real systems
%through analysis of empirically observed behavior of certain ``litmus tests''
%\cite{mador2012axiomatic,sarkar2012synchronising}. There has also been some
%work in providing operational semantics that describes the observed behavior of
%real hardware \cite{sarkar2011understanding}. Finally, there has been work
%formally relating the axiomatic specification of TSO to operational semantics\cite{x86tsocacm10}.

%The plethora of work in the area of memory models indeed suggests that this is
%a complex problem, exacerbated by the hardware architect's desire to increase
%hardware performance at the cost of complicating the memory models further. There
%are two aspects where our work can help in tackling this complex problem: (1) we
%are proposing a modular approach for verifying systems in which the components,
%namely the processors and the memory, can be verified against the memory model
%contract independent of each other, and (2) we are also proposing a design
%in which a system obeying sequential consistency is made into a high
%performance machine through the use of speculative loads followed by
%verification. The only downside is waiting for store acknowledgements, but this
%wait does not preclude the machine from performing further computations for
%future instructions without actually committing them. We are hoping to develop
%both these ideas -- of extending our proof approach for more complex memory
%models and designing real hardware which is sequentially consistent but has
%high performance.
%

%BELOW IS UNEDITTED FROM OLD PAPER


%These are lifted as is from that guy. Need to figure out how to restate them, and make it more concise.
%
%For verification of high level specifications, modern industrial practice
%consists of modeling small instances of the protocols, e.g., three CPUs
%handling two addresses and one bit of data, in terms of interleaving atomic
%steps, in guard/action languages such as Murphi [50] or TLA+ [84], and
%exploring the reachable states through explicit state enumeration.
%Symbolic methods, e.g., BDD [31] or SAT [129, 142], are yet to succeed for the
%verification of cache coherence protocols.
%Monolithic formal verification methods – methods that treat the protocol as a whole – have
%been used fairly routinely for verifying cache coherence protocols from the early 1990s
%[10 (2 nodes, murphi)]. However, these monolithic techniques will not be able to handle the very
%large state space of hierarchical protocols. Compositional verification techniques are essential. 
%
%For coherence protocols which only consider one level (or nonhierarchical protocols),
%various techniques have been proposed for the verification. These techniques basically
%include explicit state enumeration [133] and symbolic model checking [102]
%Opti-
%mizations on these techniques include partial order reduction [24], symmetry reduction
%[25, 72], compositional reasoning [96], predicate abstraction [22, 49, 82], etc. There is
%also a large body of work on parameterized verification [42, 53, 54, 100, 117]. However,
%none of them have shown the capability to verify hierarchical coherence protocols with
%realistic complexity.
%
%McMillan et al. [97, 102] modeled a 2-level MSI coherence protocol based on the
%Gigamax distributed multiprocessor [102]. In the protocol, bus snooping is used in both
%levels. SMV [97] was used to verify this protocol with two clusters each having six
%processors, for both safety and liveness properties. In terms of the complexity, many
%realistic details are abstracted away from the protocol so that the protocol does not have
%the typical complexity that a hierarchical coherence protocol has.
%
%=======
%Theorem provers have also been used in verifying cache coherence protocols,
%\eg{} FLASH~\cite{park} and Cachet~\cite{Shen:CRF, StoyShenArvind:Proofs} have
%been proved using the PVS theorem prover. Both works prove correctness of a
%single-level hierarchy while our work proves it for a multi-level hierarchy.
%The work by Park \etal{} on FLASH proves a mapping, called ``aggregation of
%transactions'', from multiple localized transactions to a single global
%transaction with PVS, and uses model-checking to prove the correctness of the
%global transactions model. However, they do not provide a methodology to prove
%the aggregation of transactions for other protocols, which we believe to be very
%difficult; in contrast, our modular proof relies on generic invariants making it
%more naturally adaptable to different protocols.
%>>>>>>> 96e9880a190d9acb6201b10e2b105d18c72f56aa

