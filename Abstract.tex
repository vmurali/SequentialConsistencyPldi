%\xxx{\textbf{Murali's version in red:} We provide a framework for modular verification of multi-processor hardware
%systems with respect to memory model specification. In this framework, we make
%use of a key observation that eventhough different memory models have different
%overall system behaviors, they all share a coherent cache hierarchy. We prove
%that this coherent cache hierarchy implements the \emph{store atomicity}
%property, which makes it behave like an atomic instantaneous memory with
%respect to the rest of the system. We further modularize the proof by defining
%what is meant for a processor to be correct independently of the memory system.
%As a concrete instance of our framework, we walk through a modular proof of
%high-performance multi-processor implementation of Lamport's Sequential
%Consistency model. The processors employ \emph{speculative execution} to
%generate many concurrent loads to the memory sub-system, while obeying our
%correctness criteria. We offer the first formal proof that when an atomic
%memory is connected to $n$ speculative processors, the system as a whole is
%sequentially consistent.  Additionally, the cache-coherent memory we present in
%this paper is the first fully verified system that supports arbitrary cache
%hierarchies. Our framework involves describing all our components using labeled
%transition systems and we machine-verify all our proofs using the Coq proof
%assistant.}
We present a new framework for modular verification of hardware designs
in the style of the Bluespec language.  That is, we formalize the idea of
components in a hardware design, with well-defined input and output channels;
and we show how to specify and verify components individually, with
machine-checked proofs in the Coq proof assistant.  In doing so, we import
to the world of hardware those modular reasoning principles that are
well-known in software verification.  As a demonstration, we verify a
fairly realistic implementation of a multicore shared-memory system.  The
main modularity split is between a memory-system component and a processor
component.  Both include nontrivial optimizations, with the memory system
employing an arbitrary hierarchy of cache nodes that communicate with each
other concurrently, and with the processor doing speculative execution of
many concurrent read operations.  Nonetheless, we prove that the combined
system implements sequential consistency.  One interesting aspect of the proof
is that the memory component is well-specified to be reused for systems
implementing a variety of weak memory models.  To our knowledge, our proof
of that memory system is the first machine verification of a cache-coherence
protocol parameterized over an arbitrary cache hierarchy, and our full-system
proof is the first machine verification of sequential consistency for a
multicore hardware design that includes memory and processors.
