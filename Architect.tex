\section{Architectural Implications}\label{sec:implications}

We have demonstrated a high-performance implementation of SC with
a distributed shared-memory system and speculative processors. However, here
are some of the concerns that an architect of a high-performance multi-processor
may have about our design:

\begin{enumerate}
\item Splitting each load into a speculative load and a verification
load increases the number of requests to the L1 cache. While verification loads
are extremely likely to hit in a cache and therefore have low latency,
nevertheless they fundamentally increase cache pressure on L1. Can we
eliminate some of this verification traffic?

\item The correctness of our model with respect to sequential consistency
crucially relies on the fact that a commit operation for only one load or store
can be active per processor at a time. Can we pipeline commit requests, \ie{}
have several of them issued at a time?

\item In some designs, a processor does not receive an explicit acknowledgement
for a store operation. Intuitively the store acknowledgment means that a store
has been placed in a global order of all stores being issued to the memory, and
this condition can potentially be determined even before the data is stored in
any cache. Can this optimization be incorporated in our framework?
\end{enumerate}

A trick architects use to reduce the need for verification loads is to use the
invalidation signal received by L1 (\eg{} in the MIPS R10000 processor~\cite{yeager1996mips}) 
to invalidate speculative loads for that address in the reorder buffer. 
The reason this works is that such a signal is an
indicator of the fact that another processor has changed the value of that
address before the load has been committed. This scheme is simultaneously
incomplete, in that it does not alleviate the need for verification loads in the
context of address speculation; and conservative, in that it invalidates loads
whose values may not have changed in spite of a write into the corresponding
cache line. Incorporating this solution will require a very small change in our
processor interface, \viz{} that of including invalidation signals from L1
caches.

The solutions for the remaining two issues require making some extra
assumptions about the memory interface. These assumptions can be enforced by
having an extra module (Load Store Buffer, LSB) in front of the memory we have
presented. The processor can issue a commit store request to the LSB and
assume that it receives the acknowledgment straightaway, continuing to issue
further commit requests to the LSB. A verification load request in the LSB
still has to receive a response from L1. It will be sent to L1 only after all
previous commit store requests in the LSB have been sent to L1. The formal
proof of why this scheme works is beyond the scope of this paper.

A more aggressive solution to maintaining the strict load-store orderings
defined by SC is to change the memory model itself. Indeed, many processors use
a more relaxed memory model; for instance, Intel x86 and Sparc use
TSO~\cite{intel64and,weaver1994sparc}, ARM and PowerPC both use their own
relaxed memory models~\cite{grisenthwaite2009arm,power2009version}.
Two things need to be noted about all relaxed memory models. First, every SC
behavior is
always a legal behavior in a relaxed model, making our speculating processor
correct in such a setting. Second, the ISA must be augmented with explicit
memory fence instructions that prevent reorderings of memory instructions
across fences, if the application demands preserving the order. Adding explicit memory fences is generally slower than implicitly
having the restrictions within the memory model, but the belief is that
explicit fences are only needed sparingly in practice and on the whole will
give better performance than the equivalent SC system.
%\murali{Starting from "two things", the rest seems out of place}

In order to use our methodology for systems with relaxed memory models, we
would have to provide an LTS description for each model. One can potentially
adjust our processor model accordingly (for
instance, more aggressive issue of even verification loads) and add the LSB described
earlier. The details are not obvious, however. Alglave
\etal{}~\cite{alglave2012formal,alglave2014herding} provide a framework for 
axiomatic specification of memory models.  Axioms regarding relaxed memory
models have also been proposed through analysis of empirically observed behavior
of certain ``litmus tests''~\cite{mador2012axiomatic,sarkar2012synchronising}.
There has also been work in providing operational semantics that describe the
observed behavior of real hardware~\cite{sarkar2011understanding}, and proving
the equivalence between the axiomatic specification of TSO and operational
semantics of a simple in-order processor with a store buffer and instantaneous
global memory~\cite{x86tsocacm10}. However, others have suggested
simplifying memory models, in other words to implement SC in real
hardware~\cite{hill1998multiprocessors}.
